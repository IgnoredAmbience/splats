\documentclass{article}
\usepackage[margin=1in]{geometry}

\title{SpLATS Executive Summary}

\begin{document}
\maketitle

\section{Introduction}

SpLATS Lazy Automated Test System is intended for Ruby developers. Its aim is to automatically perform regression testing on Ruby programs. 
In agile development, a piece of code may be refactored multiple times; it is important to ensure that version changes do not break existing functionality. SpLATS aims to automate this process by generating tests for the one version of the code, and running it against the other version. The differences between the two are made apparent by how many tests pass. A web page can be generated to demonstrate the percentage of code covered by the generated tests.
It is anticipated that SpLATS will save time and find edge cases in the code which may not have been immediately apparent.
A subset of Ruby was specified which we call Lightweight Ruby, and SpLATS is only guaranteed to work within this subset; not all programs will be tested in their entirety.

\section{Existing Solutions}
Automatic test generators have been written for a variety of languages, including Java, Haskell and Ruby, but there are no products ready for ‘real world’ usage. 
There also exists a program for Ruby called ZenTest which analyses a programmer’s code and generates method stubs for missing tests which the programmer then needs to write. This is not the purpose of SpLATS, which instead aims to remove the need to write tests altogether. 
A paper has been written on a program called RuTeG (Ruby Test case Generator) a product which has similar aims to SpLATS, but the source code has not been published, and its methodologies are somewhat different to SpLATS.

\section{Features}
For Lightweight Ruby, SpLATS can automatically generate tests to demonstrate whether or not two versions of code exhibit the same functionality; deterministic behaviour ensures that all tests are reproducible. There is both a command line interface and a graphical interface to SpLATS, each offering different options to the user. 
There are currently three different ways SpLATS generates the tests (in the form of different search space traversal algorithms), selectable from either interface. The program was designed to allow new traversal algorithms to be easily written; users can choose more suitable means of traversing the search space if they wish.

\section{Architecture}
We used a variety of third-party libraries to provide frameworks for testing, code coverage, document generation and a graphical interface.
To run, SpLATS requires a file to generate tests for. The class contained in the file is loaded, and passed to the test controller, which initialises a test generator, and a means of traversal. The generator and traversal algorithm determine what tests are generated, and these tests are then executed on the test class by the generator. The methods called on the class under test are passed mock objects as parameters; these mock objects either take primitive values from the generator, or become new mock objects. The generator the yields these tests back to the test controller, which writes them out to file.

\section{Software Engineering Methods}
The group met twice a week, one meeting was directly followed by a meeting with our supervisors. The group meetings were used to establish what each member would be doing for the week and members assisting each other with issues. They were also used to make decisions on how the project would work.
For group collaboration and version control we used Git. We used Ruby’s built in testing framework (Test::Unit) both for our own tests of our own code, and to run the tests we generated.

\section{Technical Challenges}
The biggest obstacle is a pathological one; exponential growth of the search space makes it harder to detect the useful tests that have been generated, and the sheer number of generated tests is unwieldy. To run all of the tests generated can exhaust the memory of a modest machine, even when search parameters are well chosen, but to filter out the most ‘useful’ tests is an even harder problem than generating the tests, and is beyond the scope of the project.
The other major issue was related to typing. The mock object must be able to stand in for any other object making it difficult to realise the mock object as something concrete, or containing ‘useful’ information. This was ultimately resolved by rebuilding it in each test and utilising a lookup table containing sample values for common types.

\section{Evaluation}
SpLATS has succeeded in passing its minimum requirements and many of its extensions. SpLATS generated tests which achieved 100\% code coverage on some simpler examples, even with only a search depth of two. For more complicated programs the depth becomes the limiting factor; a depth of three will test a large proportion of methods in a class but the exponential growth of the search space usually means that at depths of four or greater, either too many tests are produced to run the code without exhausting RAM, or it takes too long to generate the tests. In some programs it is also not possible to achieve 100\% coverage because language features are used that are not a part of Lightweight Ruby. The search space issue is an open problem in academia; were it solved, SpLATS would work considerably better.

\end{document}
