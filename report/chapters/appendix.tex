\appendix
\chapter{Project Proposal}
\label{appx:proposal}
\textbf{Title:} Automated lazy testing for JavaScript (or Ruby or Python or
Dart)

\textbf{Proposer:} Susan Eisenbach

Good software engineers build large test suites as they develop their code (or before) in order to ensure that they deliver bug free code.
But writing tests is a time consuming task and it is always frustrating to spend significant time in development that is not visible to your user.
The solution: automatically generate the tests.
This is easier said than done, and Tristan Allwood has done a PhD in generating tests for Haskell programs and has also worked on automatic test suite generation for Java.
This project comes from him and he would provide technical supervision for it.

There could be an interesting project in automated lazy testing for
JavaScript or Ruby / Python (i.e. dynamic, structural languages).

So:

Fully automated regression test suite generation for
JavaScript/Ruby/Python.

For example, given:
\begin{verbatim}
class LinkedList
  ... standard linked list implementation ...

  def add(param)
  ...
  end

  def size
  ...
  end

  ... etc ..
end
\end{verbatim}

Then \emph{magic happens} and you get:

\begin{verbatim}
unit test 1:
  l = new LinkedList
  assert (l.size == 0)

unit test 2:
  l = new LinkedList
  l.add(true)
  assert (l.size == 1)

unit test 3:
  l = new LinkedList
  l.add(true)
  i = l.iterator
  assert (i.hasNext)

...
\end{verbatim}

(where the unit tests are human readable, executable code using an
appropriate unit testing framework for the language)

Given a class definition (or prototype in JavaScript), the aim of this
project would be to completely automatically generate a set of unit
tests that specify the behaviour of the class, that could be suitable
for use in incremental regression testing.

There are many challenges to this project, and several ways of
approaching the problems that arise, for example:

\begin{itemize}
\item Constructors and methods in the class will need objects or primitive
  values passing in as parameters. These objects/values will need
  to be created, and should be suitable for testing.

\item For some primitives, e.g. Integer values, there could be an exponential
  blow up of meaningless testing

\item For object values, how do you create the instances?
\begin{itemize}
\item For example, you could try to guess which class is supposed to be
      passed in as an argument, and create an instance of that.
\item In Java and Haskell, we have been exploring a ``lazy
      instantiation'' technique, whereby dummy objects are passed in,
      and only as their methods are actually invoked do we generate
      test data for their return values. Would this work for dynamic
      languages?
\end{itemize}

\item Generating unit tests will require establishing what should be the
      assertions of the unit test, and capturing those values to be put
      into unit tests.
      Experimental evaluation, how do you know how good your tests are?

\item Can you find bugs between different versions of existing libraries?

\item Can you consistently achieve high code coverage of the classes
      being tested?
\end{itemize}

\chapter{Input/Output Samples}
\lstset{language=Ruby}

All the following examples show output when run with depth-limited search at a
depth of 2, to limit the number of tests shown.

\section{Simple Addition}
\subsection{Input}
\begin{lstlisting}
class Addition
  def add(a)
    1 + a
  end
end
\end{lstlisting}

\subsection{Test Output}
\lstinputlisting{listings/test_SimpleAddition.rb}

\section{A Simple Test Case}
\subsection{Input}
\lstinputlisting{listings/SimpleTest.rb}

\subsection{Output}
\lstinputlisting{listings/test_SimpleTest.rb}


