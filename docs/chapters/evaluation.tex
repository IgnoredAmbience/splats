\chapter{Validation and Conclusions}
\section{Performance of SpLATS}
  \subsection{Feature Matrix}
    \begin{tabular}{p{0.3\textwidth} | p{0.5\textwidth}}
    Aim & Success \\
    \hline
    Compare two versions of code &
    We can generate tests, which can be easily adapted to be run against another version of the code.
    No direct comparison from within the tool as yet, however. \\
    Set configuration options & FIAL \\
    Inspect code coverage of tests & FIAL \\
    Reproduce tests & DUNNO \\
    Have a CLI &
    A simple wrapper for SpLATS was implemented \\
    \hline
    Have a GUI &
    This works, and is especially useful for manual traversal \\
    Be able to be run automatically & PROBABLE WINNAR \\
    Test basic program semantics &
    It does this to a very basic level implicitly, but it's not investigated much \\
    Test code across multiple computers at once &
    As it has a CLI suitable for writing scripts around, it is possible to run multiple instances across a distributed system, for example, with Condor, however the tool doesn't distribute a single instance. \\
    \hline
    Have a website &
    This did not happen as it was less necessary than other things. \\
    Have different traversal mechaisms &
    There are 3 traversal mechanisms: depth-first, random and manual. \\
    \end{tabular}

  \subsection{Testing by Inspection}
  - Code coverage showed Tom that it wasn't doing 1 + 1 coerce error.
  \subsection{Code Coverage}
  \subsection{ZenTest}
  \subsection{Test Generation and Comparison}
    \subsubsection{Analysis}
    \subsubsection{Design}
    \subsubsection{Implementation}
    \subsubsection{Testing}
