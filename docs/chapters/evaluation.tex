\chapter{Validation and Conclusions}
\section{Performance of SpLATS}
  \subsection{Minimum Requirements}
    \begin{tabular}{|p{0.5\textwidth}|p{0.5\textwidth}|}
    \hline
    \textbf{Aim} & \textbf{Progress} \\
    \hline
    Find out if two versions of my code have the same functionality. &
    SpLATS automatically performs regression testing on a code base. \\
    \hline
    Set configuration options to select which classes and/or methods to test. &
    This requirement makes no sense in lieu of Lightweight Ruby (each file is considered a class, and the methods are loaded at run-time). \\
    \hline
    Inspect code coverage of generated tests. &
    Code coverage is automatically given to the user in a nice format when SpLATS has finished running. \\
    \hline
    Reproduce tests. &
    SpLATS behaves in a deterministic way, so the same traversal method with the same options will produce the same output. Even when random is chosen as the traversal method, there is a seed that can be used to create the same tests. \\
    \hline
    Use a command line interface to interact with the system &
    A fully functioning CLI has been written. \\
    \hline
    \end{tabular}
  \subsection{Extensions}
    \begin{tabular}{|p{0.5\textwidth}|p{0.5\textwidth}|}
    \hline
    \textbf{Aim} & \textbf{Progress} \\
    \hline
    Use a graphical interface to interact with the system &
    A fully functional GUI has been written which displays graphs to the user. \\
    \hline
    Automatically run regression tests on two versions of the code under version control. &
    We can use the standard Git tools to achieve this result. (Git bisect) \\
    \hline
    Test basic program semantics &
    We haven't considered this as we could find no reliable easily-accessible documentation for contracts which are always guaranteed to hold. \\
    \hline
    Test my code across multiple computers at once &
    We haven't considered this, although it would have been useful for the evaluation section for tests with large amounts of branches. \\
    \hline
    \end{tabular}
  \subsection{Future Enhancements}
    This table has been modified to reflect future enhancements that we have considered as the project has progressed.
    
    \begin{tabular}{|p{0.5\textwidth}|p{0.5\textwidth}|}
    \hline
    \textbf{Aim} & \textbf{Progress} \\
    \hline
    Use a website to interact with the testing &
    We haven't considered this as the GUI proved to be more challenging. \\
    \hline
    Have additional options to generate the tests, for example breadth-first and random-directed &
    We have implemented this fully. \\
    \hline
    Better traversal and pruning methods &
    This would be useful to generate tests quicker without using so much memory. \\
    \hline
    \end{tabular}

  \subsection{Evaluation of Quality of Generated Tests}
    Several sample CUTs are used to evaluate and test the quality of SpLATS'
test generation. We will also compare the ability of the depth-limited search to
reach full coverage to the random traversal strategy.

    The test classes used were:
    \begin{itemize}
      \item \textbf{Addition}: 1 plus the argument, used to test Ruby's arithmetic
internals
      \item \textbf{SimpleTest}: Tests arbitrary methods called upon arguments to the
function, branches based upon the result of the method call
      \item \textbf{LinkedList}: A standard LinkedList implementation, uses both
iterative and recursive algorithms
      \item \textbf{QuickSort}: A recursive quick sort implementation
      \item \textbf{ISBN-Tools}: An open-source library to check the validity of ISBN
numbers, makes heavy use of regular expressions and string methods
    \end{itemize}

  \subsubsection{Result}

    \begin{tabular}{|c|c|c|c|c|c|c|c|}
\hline
\textbf{CUT} & \textbf{Search Depth} & \textbf{Generation Time/m:s} & \textbf{Tests} & \textbf{Coverage} \\
& \textbf{(Mock Depth)} & & & \\
\hline
Addition & 2 & 0:0.051 & 6 & 3/3 (100\%) \\
\hline
SimpleTest & 2 & 0:0.064 & 32 & 12/12 (100\%) \\
\hline
LinkedList & 2 & 0:0.056 & 31 & 35/47 (74\%) \\
& 3 & 0:0.147 & 306 & 45/47 (96\%) \\
& 4 & 0:2.118 & 3859 & 46/47 (98\%)\\
\hline
QuickSort & 2 (1) & 0:0.344 & 87 & 8/8 (100\%) \\
& 2 (2) & 0:3.356 & 4272 & 8/8 (100\%) \\
& 2 (3) & 5:57.2 & 351927 & Not testable\\
\hline
ISBN-Tools & 2 (1) & 0:0.056 & 13 & 10/16 (63\%) \\
& 2 (2) & 0:0.072 & 57 & 10/16 (63\%) \\
& 2 (3) & Stopped after 30:00 & 341472 & Not testable \\
\hline
\end{tabular}

\subsubsection{Analysis}
  The simplest classes (Addition and SimpleTest) easily achieved 100\% coverage at the minimum reasonable depth of 2.
This is because their output values are not dependent upon the object's internal state so every execution path can be reached from the first method called after the constructor.

  LinkedList is a more interesting example as its internal state determines the output.
It can be seen that at a depth of 2, only three-quarters of the lines can be reached as the majority of the execution paths depend on a list containing more than one element, which is only just possible at a depth of 2.

  At a depth of 3 almost all of the lines are accessible to the tests, except for two.
The first of these is only reachable in a list of more than two elements, again not possible at this traversal depth.
This is possible at a depth of 4.

  The remaining uncovered line is never reached because it is immediately preceded by a line that is outside of our Lightweight Ruby specification.
This line always raises an Exception, (due to trying to yield to a non-existant block) meaning that the uncovered line is unreachable by our test generator.

  It is quite noticable that QuickSort and ISBN-Tools are less suited to a complete traversal method.
Similarly to the Addition and SimpleTest classes, the methods on these classes do not depend on the internal state of the object to produce their output.
The traversal depth (the number of TestLines produced) does not impact upon the usefulness of the tests produced.

  However, it can be seen that the permitted upper bound on the length of chains of Mock objects directly impacts upon the number and complexity of tests that are generated.
This is because each call on a mock object is treated as splitting the search space.
As both of these classes call many methods upon the mock objects, QuickSort through recursion, and ISBN-Tools by repeatedly calling a long chain of methods.

\section{Testing SpLATS}
\subsection{Testing the Core}
  As this project was heavily research-based, it was unfeasable to write tests before implementing the code as the code underwent frequent and radical refactorings or change of scope.
  Tests were therefore written for a section of code once it reached a stable point and was considered ready to reach the final product.
  Code that was in the master branch of the repository before this rule was enforced has since had unit tests written to cover it.
    
\subsection{Testing of GUI}
The underlying code of SpLATS has been tested, and so it is assumed that the GUI will work provided all the buttons do what they say they will.
A thorough run-through of the GUI was performed with complicated and simple execution paths. Some errors were found, and these were easily uncovered and easily fixed.
There are a few libraries which exist that can be used to automatically test Ruby GUIs, however, with so few buttons and a very limited range of behaviours, these were deemed too complicated to learn for very little benefit.
