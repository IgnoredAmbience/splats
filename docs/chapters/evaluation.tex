\chapter{Validation and Conclusions}
\section{Performance of SpLATS}
  \subsection{Minimum Requirements}
    \begin{tabular}{| c | p{\colwidth} |}
    \hline
    \textbf{Aim} & \textbf{Progress} \\
    \hline
    Find out if two versions of my code have the same functionality. &
    SpLATS automatically performs regression testing on a code base. \\
    \hline
    Set configuration options to select which classes and/or methods to test. &
    This requirement makes no sense in lieu of Lightweight Ruby (each file is considered a class, and the methods are loaded at run-time). \\
    \hline
    Inspect code coverage of generated tests. &
    Code coverage is automatically given to the user in a nice format when SpLATS has finished running. \\
    \hline
    Reproduce tests. &
    SpLATS behaves in a deterministic way, so the same traversal method with the same options will produce the same output. Even when random is chosen as the traversal method, there is a seed that can be used to create the same tests. \\
    \hline
    Use a command line interface to interact with the system &
    A fully functioning CLI has been written. \\
    \hline
    \end{tabular}
  \subsection{Extensions}
    \begin{tabular}{| c | p{\colwidth} | p{\colwidth} |}
    \hline
    \textbf{Aim} & \textbf{Progress} \\
    \hline
    Use a graphical interface to interact with the system &
    A fully functional GUI has been written which displays graphs to the user. \\
    \hline
    Automatically run regression tests on two versions of the code under version control. &
    We can use the standard Git tools to achieve this result. (Git bisect) \\
    \hline
    Test basic program semantics &
    We haven't considered this as we could find no reliable easily-accessible documentation for contracts which are always guaranteed to hold. \\
    \hline
    Test my code across multiple computers at once &
    We haven't considered this, although it would have been useful for the evaluation section for tests with large amounts of branches. \\
    \hline
    \end{tabular}
  \subsection{Future Enhancements}
    This table has been modified to reflect future enhancements that we have considered as the project has progressed.
    \begin{tabular}{| c | p{\colwidth} | p{\colwidth} |}
    \hline
    \textbf{Aim} & \textbf{Progress} \\
    \hline
    Use a website to interact with the testing &
    We haven't considered this as the GUI proved to be more challenging.
    \hline
    Have additional options to generate the tests, for example breadth-first and random-directed &
    We have implemented this fully. \\
    \hline
    Better traversal and pruning methods &
    This would be useful to generate tests quicker without using so much memory. \\
    \hline
    \end{tabular}

  \subsection{Testing by Inspection}
  - Code coverage showed Tom that it wasn't doing 1 + 1 coerce error.
  \subsection{Code Coverage}
  \subsection{ZenTest}
  \subsection{Test Generation and Comparison}
    \subsubsection{Analysis}
    \subsubsection{Design}
    \subsubsection{Implementation}
    \subsubsection{Testing}
