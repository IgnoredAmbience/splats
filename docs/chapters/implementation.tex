\chapter{Implementation}
\section{Technical Challenges}
  \subsection{Duck Typing}
    Duck typing is a method of dynamic typing, named after the phrase "if it looks like a duck and quacks like a duck, it must be a duck".
    Instead of a function requiring parameters to be of specific types, or implementing specific interfaces, any object can have any method called on it.
    If the method is not defined for that object then a runtime error will occur.
    Because of this loose typing mechanism, it's very easy to make a Mock object that can take the place of any other class of object, by responding to all methods, returning another Mock object if output is required.
    It can subsequently collect all the required methods a given parameter is needed to respond to, building up information about what parameters for a particular code-flow need to be.

    However, this also makes it very difficult to output reusable tests.
    Given duck typing doesn't lead to explicit types or interfaces, it's difficult to impossible to write an explicit `concrete' output that can be recreated from scratch without having to include the entire execution tree.\\
    TODO: HOW THIS WAS RESOLVED \\

  \subsection{GUI}
    Shoes ain't made for walking, but it's still gonna walk right over you.
