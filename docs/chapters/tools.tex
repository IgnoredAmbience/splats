\chapter{Toolage}

\section{Collaboration and Coordination}

  \subsection{Meeting Styles}
    The group first tried short, daily meetings, with longer meetings where possible or necessary.
    We quickly found that conflicting schedules made the short meetings impossible for all group members to attend and served little purpose, most days not enough progress was made to warrant discussion.
    Daily meetings would be useful in a workplace scenario where considerable time is spent every day and team members will all be present, quite different to the timetable of our group who combined covered seven lecture series and two humanities options.

    Instead, there were two hour long meetings per week, one directly followed by a meeting with our supervisor.

    After trying several online communication methods, we found a project IRC channel was helpful, hosted on EsperNet\footnote{\url{http://esper.net}}.
    Commit messages to the were posted there, and it was especially useful for discussions during the winter vacation, when we couldn't be in the same physical location.
    It's also easy to log everything said for reference or filling in members not present.

    Commit messages were also posted to a twitter account, which was useful for the members of the group that use the service.


\section{Development Methods}

  \subsection{Working Styles}
    We tried a number of styles, including individual and pair work, and a long group hack session.

    The hack session worked best.
    Should have done more of them.
    Damnit.

  \subsection{Tools Used}
    \subsubsection{Code Control}
      We decided at the outset to use the git\footnote{\url{http://git-scm.com}} revision control system.
      We had all used it in the past, and were comfortable with it.
      It is a distributed revision control system, which grants additional flexibility compared to centralized systems, as it allows for such things as offline working and private branches.

      For where to host the central repository, we had 3 main options: github\footnote{\url{http://github.com}}, bitbucket\footnote{\url{http://bitbucket.org}} and the department's own servers.
      Some of us already had github accounts and workflows set up for it, however private repositories are a paid-only feature.
      The department's own servers have good uptime, but have been known to become inaccessable at short notice, and they don't offer as many options or extra features as the others.
      Bitbucket allow private repositories for unpaid accounts, and have other additional extras, such as a project wiki and bug tracking system.

      As such, we decided to use bitbucket.
      For the vast majority of the project, it seemed a good choice, stable and available, with the wiki being useful.
      However, shortly before the deadline for this report, it went down, and we had to switch to a backup repo on the Department of Computing servers.

    \subsection{Documentation Control}
      For writing reports, we primarily used two systems for both writing and revision control.

      Initial drafts were done in Google Docs\footnote{\url{http://docs.google.com}}.
      In addition to a simple WSYWIG interface, it allows for real-time simultaneous editing and collaboration, as well as rudimentary revision control, and as such is well suited to rapid prototyping of a document.

      Final drafts were done in \LaTeX\footnote{\url{http://latex-project.org}}, a document markup language and typesetter.
      \LaTeX is more complex to use than Google Docs, being a programming language in its own right, however it produces `prettier' output.
      Being written as plain text, it is easiest to track changes using our existing git setup.

  \subsection{Code Style Guidelines}
    \begin{itemize}
      \item Standard Ruby naming conventions
    \end{itemize}


\section{Summary of team member contributions}
% http://www.doc.ic.ac.uk/~tw1509/splatstats/
Disclaimer: not all commits are created equal

  \subsection{Ethel}
    Fiddly bits around the edges.

    First started with writing the output module, to turn symbolic tests to actual ruby code to be run later.

    Wrote the bulk of the command-line interface.

    Also tweaked and documented various other bits of code.
    \begin{description}
      \item[Commits] 70
      \item[Lines]
    \end{description}

  \subsection{Caz}
    Secretary, report writing, core work
    \begin{description}
      \item[Commits] 20
      \item[Lines]
    \end{description}

  \subsection{Chris}
    Stuff on core, outer core
    \begin{description}
      \item[Commits] 21
      \item[Lines]
    \end{description}

  \subsection{Joe}
    Freeloader in chief
    \begin{description}
      \item[Commits] 13
      \item[Lines]
    \end{description}

  \subsection{Tom}
    Lead developer, lots on the core
    \begin{description}
      \item[Commits] 74
      \item[Lines]
    \end{description}
