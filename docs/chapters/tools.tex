\chapter{Toolage}

\section{Collaboration and Coordination}

  \subsection{Meeting Styles}
    The group first tried short, daily meetings, with longer meetings where possible or necessary.
    We quickly found that conflicting schedules made the short meetings impossible for all group members to attend and served little purpose, most days not enough progress was made to warrant discussion.
    Daily meetings would be useful in a workplace scenario where considerable time is spent every day and team members will all be present, quite different to the timetable of our group who combined covered seven lecture series and two humanities options.

    Instead, there were two hour long meetings per week, one directly followed by a meeting with our supervisor.

    After trying several online communication methods, we found a project IRC channel was helpful, hosted on EsperNet\footnote{\url{http://esper.net}}.
    Commit messages from Git were posted there, and it was especially useful for discussions during the winter holidays, when we couldn't be in the same physical location. The conversations are also automatically logged, so members who were not present in the chat room at the time could catch up.

    Commit messages were also posted to a twitter account and mobile phones, useful for the members of the group who use the services.

\section{Development Methods}

  \subsection{Working Styles}
  The majority of the time, the big group meetings would highlight areas of the code or reports which needed working on, ways to resolve the issues would be discussed, and it would be decided to be either a single, pair or group programming session. The group members tend to think about problems in lots of different ways, which was helpful at the early stages of the project, but became more difficult as concrete code needed to be written. Paired programming quickly proved to not work very well, as every member of the group was used to working individually. Paired programming was useful in ten minute chunks, where someone was very stuck and a fresh look helped.
  A few long sessions were had, where the group members met and sat in a room programming for 12 hours. This helped to keep people focussed, and also to ensure that enough breaks were taken. Unfortunately, due to the conflicting timetables, we were unable to have as many of these sessions as we would have liked, as they worked extremely well and were definitely the most productive.

  \subsection{Tools Used}
    \subsubsection{Version Control}
      We decided at the outset to use the git\footnote{\url{http://git-scm.com}} revision control system for various reasons:
      \begin{itemize}
      \item We had all used it in the past and were comfortable with it.
      \item It is a distributed revision control system, which grants additional flexibility compared to centralized systems, as it allows for such things as offline working and private branches.
      \item It works very well with Unix systems, and all but one of the group runs Unix.
      \item Git handles complex merging extremely well, and when there is a merge conflict, it is very easy to sort manually.
      \end{itemize}

      For where to host the central repository, we had 3 main options: github\footnote{\url{http://github.com}}, bitbucket\footnote{\url{http://bitbucket.org}} and the department's own servers.
      Some of us already had github accounts and workflows set up for it, however private repositories are a paid-only feature.
      The department's own servers have good uptime, but have been known to become inaccessable at short notice, and they don't offer as many options or extra features as the others.
      Bitbucket allow private repositories for unpaid accounts, and have other additional extras, such as a project wiki and bug tracking system.

      As such, we decided to use bitbucket.
      For the vast majority of the project, it seemed a good choice, stable and available, with the wiki being useful.
      However, shortly before the deadline for this report, it went down, and we had to switch to a backup repo on the Department of Computing servers.

    \subsection{Documentation Control}
      For writing reports, we primarily used two systems for both writing and revision control.

      Initial drafts were done in Google Docs\footnote{\url{http://docs.google.com}}.
      In addition to a simple WSYWIG interface, it allows for real-time simultaneous editing and collaboration, as well as rudimentary revision control, and as such is well suited to rapid prototyping of a document.

      Final drafts were done in \LaTeX\footnote{\url{http://latex-project.org}}, a document markup language and typesetter.
      \LaTeX is more complex to use than Google Docs, but produces `prettier' output. As Latex is written in plain text, this also means that it's easy to track changes using the existing git setup.

  \subsection{Code Style Guidelines}
  We followed the standard Ruby naming conventions
    \begin{itemize}
      \item All method, symbol and variable names are wholly lowercase with words seperated by underscores e.g. 'this\_is\_a\_variable'
      \item All class names are capitalised camel case e.g 'ArbitraryClass'
      \item All constants are wholly uppercase with words seperated by underscores e.g 'MAX\_SIZE'
    \end{itemize}


\section{Summary of team member contributions}
% http://www.doc.ic.ac.uk/~tw1509/splatstats/
  A brief summary of each member's contributions to the project, complete with the number of commits and lines changed.

  Disclaimer: not all commits are created equal

  \subsection{Ethel}
    Mostly forwent work on the core, instead working on fiddly bits around the edges.

    First started with writing the output module in its various stages.

    Wrote a revision of the command-line interface.

    Also tweaked and documented various other bits of code.

    Toward the end, lots on the report.
    \begin{description}
      \item[Commits] 70
      \item[Lines]
    \end{description}

  \subsection{Caz}
    Secretary, report writing, core work
    \begin{description}
      \item[Commits] 20
      \item[Lines]
    \end{description}

  \subsection{Chris}
    Stuff on core, outer core
    \begin{description}
      \item[Commits] 21
      \item[Lines]
    \end{description}

  \subsection{Joe}
    Freeloader in chief
    \begin{description}
      \item[Commits] 13
      \item[Lines]
    \end{description}

  \subsection{Tom}
    Lead developer, lots on the core
    \begin{description}
      \item[Commits] 74
      \item[Lines]
    \end{description}
