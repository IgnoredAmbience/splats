\chapter{Design}

\section{Choice of Language}
  Initially the project offered a choice of three languages to target: JavaScript, Python or Ruby.
  These languages are all duck typed\footnote{If it quacks like a duck, it's a duck}, thus all introducing similar issues with automatically generating tests, namely that the finished code must be `concrete'.
  All three allow object-oriented approaches, but Ruby is the only one which is purely object-oriented, the other two having `raw' types, such as integers.
  Python and Javascript therefore would require distinguishing between the two classes of variable, whereas Ruby would be easier to deal with.

  One downside of Ruby is that none of us had used it much before.
  However, it was easy enough to pick up, so this was not much of a problem in the end.

\section{Choice of Test Method to Implement}
  There are many types of tests we could have produced with SpLATS.
  The two main categories are whitebox\footnote{Testing the components} and blackbox\footnote{Testing the application as a whole} testing.
  We decided to choose blackbox testing with the option of implementing some whitebox testing as an extension.
  Because Ruby is dynamically typed, the types of whitebox testing which can be implemented are limited to generic conditions.
  [examples here]

\section{Existing Products}
  \subsection{Selenium}
    Selenium enables testing web page for Ruby, but we wanted this project to be more generalised than that.

  \subsection{Test::Unit}
    Ruby comes shipped with this module\footnote{\url{http://www.ruby-doc.org/stdlib-1.9.2/libdoc/test/unit/rdoc/Test/Unit.html}} which offers to programmers the ability to write test suites, tests and cases.
    The tests are not automatically produced, and are not automatically run on later versions of the code.

  \subsection{ZenTest}
    ZenTest\footnote{\url{http://docs.seattlerb.org/ZenTest/index.html}} uses Ruby's Test::Unit to automatically test code, both blackbox and whitebox.
    As a file is saved, ZenTest automatically runs itself again.
    ZenTest uses code coverage to highlight tests which need to be written by the developer, and generates stubs for these.
    ZenTest also uses whitebox testing to check if a method has been implemented or not.
    Whilst this tool is extremely useful, it still expects developers to write the tests themselves.
    The aim of SpLATS is to focus the attention away from writing tests, and to allow a developer to just code.
    The aim of ZenTest is to make test writing as easy as possible.

  \subsection{IRULAN}
    IRULAN\footnote{\url{http://www.doc.ic.ac.uk/~tora/thesis.pdf}} is an automated blackbox testing system for Haskell written by Dr Allwood as his PhD thesis.
    IRULAN performs directed random testing, which is a different method to how SpLATS performs (see below).

  \subsection{Tickling Java with a Feather}
    This\footnote{\url{http://pubs.doc.ic.ac.uk/testing-Java-with-fj/testing-Java-with-fj.pdf}} is another of Dr Allwood's research papers, where he performs automated testing on a subset of the Java language.
    Some of the methods outlined here are useful.

\section{Features}
  Below is an extract from the first report that was written explaining what the requirements are.
  These have been modified slightly to take into consideration how the requirements have changed as the project has evolved.
  \subsection{Minimum Requirements}
    \begin{tabular}{| c | p{5cm} | p{5cm} |}
    \hline
    \textbf{As a...} & \textbf{I need to...} & \textbf{Because...} \\
    \hline
    Ruby developer & find out if two versions of my code have the same functionality & any changes to the codebase should only affect a specific subset of the program’s output. \\
    \hline
    Ruby developer & set configuration options to select which classes and/or methods to test & I won't necessarily want to test the entire system \\
    \hline
    Ruby developer & specify how the tests should be generated (human-directed, random or depth-first limited) & different traversal methods give different results of tests produced \\
    Ruby developer & inspect code coverage of generated tests & I need to know how much of my code has been tested \\
    \hline
    Ruby developer & reproduce tests & a test will highlight issues, I will change the code and then want to run the same tests to check the problem has disappeared. \\
    \hline
    Ruby developer & use a Command Line Interface to interact with the system & it's the quickest method for me to run tests \\
    \hline
    \end{tabular}
  \subsection{Extensions}
    \begin{tabular}{| c | p{5cm} | p{5cm} |}
    \hline
    \textbf{As a...} & \textbf{I need to...} & \textbf{Because...} \\
    \hline
    Ruby developer & use a Graphical User Interface to interact with the system & if I'm not used to Command Line Interfaces, it will be the easiest way for me to interact. \\
    \hline
    Ruby developer & automatically run regression tests on two versions of the code under version control & I won't have to specify where the versions are, I can just run a command telling it to use references in version control \\
    \hline
    Ruby developer & test basic program semantics (whitebox testing) & two versions of the same code may have the same functionality, but they could be fundamentally wrong \\
    \hline
    Ruby developer & test my code on a distributed system & parallelisation would allow for more tests to be created in a quicker amount of time, and the processing power could enable more tests to be produced which find bugs that otherwise wouldn't have been found. \\
    \hline
    \end{tabular}
  \subsection{Future Enhancements}
    \begin{tabular}{| c | p{5cm} | p{5cm} |}
    \hline
    \textbf{As a...} & \textbf{I need to...} & \textbf{Because...} \\
    \hline
    Generic user & Use a website to interact with the testing & I want to learn how to test and therefore what good tests should look like. \\
    \hline
    Ruby developer & Have additional options to generate the tests, for example breadth-first and random-directed & My program may respond better to different traversal methods \\
    \hline
    \end{tabular}
  \section{Architecture}
