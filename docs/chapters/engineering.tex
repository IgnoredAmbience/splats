\chapter{Software Engineering}
\section{Technical Challenges}
\section{Collaboration and Coordination}

  \subsection{Meeting Styles}
    We started with the aim of having regular, daily meetings, in the style of scrum.
    However, conflicting timetables lead to these quickly falling apart, as, between us, we covered all but one lecture series, and thus finding times when we were all available was difficult.

    We eventually settled on a single weekly meet with our supervisors, and two group meetings a week, one before the supervisor one.

    We also found, later on, that a project IRC channel was helpful.


\section{Development Methods}

  \subsection{Working Styles}
    We tried a number of styles, including individual and pair work, and a long group hack session.

    The hack session worked best.
    Should have done more of them.
    Damnit.

  \subsection{Tools Used}
    \subsubsection{Code Control}
      We decided at the outset to use the git\footnote{\url{http://git-scm.com}} revision control system.
      We had all used it in the past, and were comfortable with it.
      It is a distributed revision control system, which grants additional flexibility compared to centralized systems, as it allows for such things as offline working and private branches.

      For where to host the central repository, we had 3 main options: github\footnote{\url{http://github.com}}, bitbucket\footnote{\url{http://bitbucket.org}} and the department's own servers.
      Some of us already had github accounts and workflows set up for it, however private repositories are a paid-only feature.
      The department's own servers have good uptime, but have been known to become inaccessable at short notice, and they don't offer as many options as the others.
      Bitbucket allow private repositories for unpaid accounts, and have other additional extras, such as a project wiki and bug tracking system.

			As such, we decided to use bitbucket.
			It wuz pretty gud, akshully.

    \subsection{Documentation Control}
      For writing reports, we primarily used two systems for both writing and revision control.

      Initial drafts were done in Google Docs\footnote{\url{http://docs.google.com}}.
      In addition to a simple WSYWIG interface, it allows for real-time simultaneous editing and collaboration, as well as rudimentary revision control, and as such is well suited to rapid prototyping of a document.

      Final drafts were done in \LaTeX\footnote{\url{http://latex-project.org}}, a document markup language and typesetter.
      \LaTeX is more complex to use than Google Docs, being a programming language in its own right, however it produces `prettier' output.
      Being written as plain text, it is easiest to track changes using our existing git setup.

  \subsection{Code Style Guidelines}
    \begin{itemize}
      \item Standard Ruby naming conventions
    \end{itemize}


\section{Summary of team member contributions}
% http://www.doc.ic.ac.uk/~tw1509/splatstats/
Disclaimer: not all commits are created equal

  \subsection{Ethel}
    Fiddly bits around the edges.

    First started with writing the output module, to turn symbolic tests to actual ruby code to be run later.

    Wrote the bulk of the command-line interface.

    Also tweaked and documented various other bits of code.
    \begin{description}
      \item[Commits] 70
      \item[Lines]
    \end{description}

  \subsection{Caz}
    Secretary, report writing, core work
    \begin{description}
      \item[Commits] 20
      \item[Lines]
    \end{description}

  \subsection{Chris}
    Stuff on core, outer core
    \begin{description}
      \item[Commits] 21
      \item[Lines]
    \end{description}

  \subsection{Joe}
    Freeloader in chief
    \begin{description}
      \item[Commits] 13
      \item[Lines]
    \end{description}

  \subsection{Tom}
    Lead developer, lots on the core
    \begin{description}
      \item[Commits] 74
      \item[Lines]
    \end{description}
