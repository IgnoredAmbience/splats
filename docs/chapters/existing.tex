\chapter{The State of the Art}

\section{Existing Products}

  \subsection{Test::Unit}
    Ruby comes shipped with this module\footnote{\url{http://www.ruby-doc.org/stdlib-1.9.2/libdoc/test/unit/rdoc/Test/Unit.html}} which offers to programmers the ability to write test suites, tests and cases.
    The tests are not automatically produced, and are not automatically run on later versions of the code.

  \subsection{ZenTest}
    ZenTest\footnote{\url{http://docs.seattlerb.org/ZenTest/index.html}} uses Ruby's Test::Unit to automatically test code, both blackbox and whitebox.
    As a file is saved, ZenTest automatically runs itself again.
    ZenTest uses code coverage to highlight tests which need to be written by the developer, and generates stubs for these.
    ZenTest also uses whitebox testing to check if a method has been implemented or not.
    Whilst this tool is extremely useful, it still expects developers to write the tests themselves.
    The aim of SpLATS is to focus the attention away from writing tests, and to allow a developer to just code.
    The aim of ZenTest is to make test writing as easy as possible.

  \subsection{IRULAN}
    IRULAN\footnote{\url{http://www.doc.ic.ac.uk/~tora/thesis.pdf}} is an automated blackbox testing system for Haskell written by Dr Allwood as his PhD thesis.
    IRULAN performs directed random testing, which is a different method to how SpLATS performs (see below).

  \subsection{Tickling Java with a Feather}
    This\footnote{\url{http://pubs.doc.ic.ac.uk/testing-Java-with-fj/testing-Java-with-fj.pdf}} is another of Dr Allwood's research papers, where he performs automated testing on a subset of the Java language.
    Some of the methods outlined here are useful.

  \subsection{Randoop}
    Randoop\footnote{\url{http://code.google.com/p/randoop/}} is another directed random testing framework, like IRULAN, but for Java.
    It generates a progressively growing sequence of method calls, before running standard tests (eg, \verb|o.equals(o)|).
    It has been used to find errors in a number of commonly used libraries.

