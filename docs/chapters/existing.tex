\chapter{Existing Products}

There are a number of existing products which perform sub-tasks of what SpLATS offers, but none of them offer a complete solution for automatic testing of Ruby code. Some of the existing products have become integral to SpLATS and others have offered ideas.

\section{Test Frameworks}

  \subsection{Test::Unit}
    The Ruby standard library includes this
module\footnote{\url{http://www.ruby-doc.org/stdlib-1.9.2/libdoc/test/unit/rdoc/Test/Unit.html}}
which is a testing framework.

Many other competing testing frameworks exist, most providing little added
benefit other than syntactical differences.

SpLATS targets Test::Unit and Minitest::Unit as its output format.

\section{Static Analysis}

  Static analysis is a form of code analysis performed without actually
  executing the code itself. Generally the code is inspected, possibly
  transformed into a simpler or more abstract state, and then inferrences
  are made about the behaviour of the program.

  For example, lint\footnote{\url{http://www.manpages.unixforum.co.uk/man-pages/unix/freebsd-6.2/1/lint-man-page.html}}
  is a tool which was originally designed to detect potential errors in
  code such as assignment instead of comparison for equality, type mismatches,
  or dereferencing of null pointers.

  Ultimately, static analysis is hard in Ruby. The nature of the language is that
  almost all behaviour can be modified at run time; without running the program,
  it is difficult to infer anything about its behaviour. Furthermore, it was beyond
  the scope of the project.

\section{Dynamic Analysis} 

  Dynamic analysis analyses code by running it, potentially with modifications
  to the runtime environment. Code profiling is an example of dynamic analysis;
  the speed and space requirements of a program are analysed by running the
  program.

  \subsection{ZenTest}
    ZenTest\footnote{\url{http://docs.seattlerb.org/ZenTest/index.html}}
comprises of a suite of 4 tools designed to assist a software developer with
writing tests for their Ruby code. The tool of particular interest within the suite is itself called ZenTest,
which aims to speed the writing of tests by inspecting the class under test (CUT)
and the test class (TC) for the CUT.

    ZenTest compares the functions defined on the CUT with the methods defined
on the TC. It then produces stub test methods on the TC for those methods on the
CUT it believes are untested.

ZenTest deduces which methods are being tested by analysing the method names; assuming that for every method, \emph{name}, on the CUT there should be a corresponding method on the TC named \texttt{test\_\emph{name}}. This is a very superficial and basic method for determining whether or not a class is completely tested.

    Whilst this tool is extremely useful, it still expects developers to write
the tests themselves. The aim of SpLATS is to focus the attention away from
writing tests, and to allow a developer to just write code.
  
  \subsection{KLEE}
  KLEE\footnote{\url{http://klee.llvm.org/}} is a symbolic virtual machine which can run on most compiled languages. It can automatically generate tests which achieve high code coverage on complex and memory-intensive programs. KLEE constructs symbolic command line strings and calls the main function with these symbols. When these arguments are called, KLEE constructs a constraint solver, forks on the options and follows all possible paths of execution. When an error can possibly occur, KLEE checks if any possible value for the symbol will cause that error. It then prunes the tree, generates concrete values, and continues following the execution path.\footnote{\url{http://www.usenix.org/event/osdi08/tech/full_papers/cadar/cadar_html/}}

  The execution path of KLEE is defined by the programmer as annotations, but the idea of SpLATS is to liberate the programmer from doing anything other than programming. Having to annotate code shifts the paradigm that the programmer is free, therefore using similar architecture or methods to KLEE was deemed unsuitable.

  \subsection{IRULAN}
    IRULAN\footnote{\url{http://www.doc.ic.ac.uk/~tora/thesis.pdf}} is an 
    automated blackbox testing system for Haskell written by Dr Allwood 
    as his PhD thesis.

    IRULAN performs directed random testing, whereas SpLATS works using a variety 
    of methods to traverse the search space.

  \subsection{Tickling Java with a Feather}
    Tickling Java with a Feather
    \footnote{\url{http://pubs.doc.ic.ac.uk/testing-Java-with-fj/testing-Java-with-fj.pdf}}
    is another of Dr Allwood's papers, where testing is restricted a well defined
    subset of the Java language. 

    From this paper, we took the idea of only testing on a subset of Ruby, and 
    simply ignoring certain other parts of the language.


  \subsection{Randoop}
    Randoop\footnote{\url{http://code.google.com/p/randoop/}} is another
feedback-directed random testing framework for Java.

    It generates a progressively growing sequence of method calls, which are
then executed and checked to see if the result is useful, redundant, illegal or
contract-violating. The test is then reused as a seed if determined to be
useful.

    Contract-violating test cases are also output as a means of highlighting
that the CUT violates a semantic rule. The standard contracts are derived from
the Java Language Specification. Further contracts can be added to the system by
the user.

    It has been used to find errors in a number of commonly used libraries.

  \subsection{RuTeG}
    RuTeG\footnote{S. Mairhofer, R. Feldt, R. Torkar, Search-based software
testing and Test Data Generation for a Dynamic Programming Language} (Ruby Test
case Generator) is a project with similar aims to SpLATS, but approaches the
problem from a slightly different angle.

RuTeG attempts to find the appropriate type of variable to pass into the CUT
methods by using a "selecting data generator" for each argument. This selection
forms a part of the search space of the test generation, which is directed using
a genetic algorithm.

RuTeG was successful in its aims to achieve higher-level coverage at a faster
rate than a random search method.
